C++ Arg\+Parse is an open source header-\/only C++ library. Due to the core\textquotesingle{}s heavy usage of template magic, the library can expose a very easy-\/to-\/use interface, so you as an application developer only have to care about the good stuff. Think Python\textquotesingle{}s \href{https://docs.python.org/3/library/argparse.html}{\tt argparse}.

\href{https://travis-ci.org/backraw/cppargparse}{\tt }

{\itshape Support for Windows is currently untested.}

If you want to skip the following example and core presentations, you can do so by going straight to \href{#Build-tests-and-install}{\tt Build tests and install}.

The source code documentation can be found here\+: \href{https://backraw.github.io/cppargparse}{\tt https\+://backraw.\+github.\+io/cppargparse}

\section*{The library interface}

Let me throw some sample code in there\+: 
\begin{DoxyCode}
\{C++\}
#include <string>
#include <vector>

#include <cppargparse/cppargparse.h>


// Struct for parsed command line data
struct CmdValues
\{
    bool t;

    int x;
    float z;
    double d;
    long double ld;

    std::string str;

    std::vector<int> ints;
    std::vector<std::string> strings;
\};



int main(int argc, char *argv[])
\{
    // Parse the command line arguments
    using namespace cppargparse;
    auto arg\_parser = parser::ArgumentParser(argc, argv);

    // for example: "-t -x 50"


    // Create an instance for our parsed command line values
    CmdValues cmdvalues;

    // Check for the flag: -t
    // True if the user pass -t, false if not
    cmdvalues.t = arg\_parser.get\_flag("-t");


    // Check for the option: -x
    cmdvalues.x = arg\_parser.get\_option<int>("-x");

    // This above statement will raise an errors::CommandLineArgumentError,
    // if the user didn't pass -t as a command line argument.
    // To get around this, you can either use a try/catch block...

    try
    \{
        cmdvalues.x = arg\_parser.get\_option<int>("-x");
    \}
    catch (const errors::CommandLineArgumentError &error)
    \{
        // do something with the error
        return -1;
    \}

    // ... or provide a default value for -x:
    cmdvalues.x = arg\_parser.get\_option<int>("-x", 0);

    // This will set cmdvalues.x to 0 if the user didn't pass -x <value>


    // Floats, doubles and long doubles are the same
    cmdvalues.z = arg\_parser.get\_option<float>("-z", 742.22f);
    cmdvalues.d = arg\_parser.get\_option<double>("-d");
    cmdvalues.ld = arg\_parser.get\_option<long double>("-ld");


    // Strings:
    cmdvalues.str = arg\_parser.get\_option<std::string>("--str", "--str NOT PASSED");


    // Vectors:
    cmdvalues.ints = arg\_parser.get\_option<std::vector<int>>("--ints", std::vector<int>());
    cmdvalues.strings = arg\_parser.get\_option<std::vector<std::string>>("--strings",
       std::vector<std::string>());


    return 0;
\}
\end{DoxyCode}


For more examples see \href{https://github.com/backraw/cppargparse/tree/master/samples}{\tt https\+://github.\+com/backraw/cppargparse/tree/master/samples}.

\section*{The core}

All the magic is done via the typed {\ttfamily \hyperlink{structcppargparse_1_1argument}{cppargparse\+::argument}} struct. Each such struct definition {\bfseries must provide 3 static methods}\+:
\begin{DoxyItemize}
\item {\ttfamily T parse(cmd, cmdarg, cmdargs)}
\item {\ttfamily T convert(cmd, cmdarg, cmdargs)}
\item {\ttfamily const char $\ast$error\+\_\+string(cmdarg)}
\end{DoxyItemize}

Parameter definition\+:
\begin{DoxyItemize}
\item {\ttfamily cmd} represents the whole command line inside a {\ttfamily std\+::vector$<$std\+::string$>$}
\item {\ttfamily cmdarg} represents the argument iterator position inside {\ttfamily cmd}
\item {\ttfamily cmdargs} represents the command line argument map, e.\+g. {\ttfamily \char`\"{}arg1\char`\"{} =$>$ iterator position 0}
\end{DoxyItemize}

The {\ttfamily \hyperlink{classcppargparse_1_1parser_1_1ArgumentParser}{cppargparse\+::parser\+::\+Argument\+Parser}} class provides the {\ttfamily get\+\_\+option$<$T$>$(arg)} and {\ttfamily get\+\_\+flag(arg)} methods for calling the actual type converter methods provided by the {\ttfamily \hyperlink{structcppargparse_1_1argument}{cppargparse\+::argument}$<$T$>$} structs.

Assume the following command line arguments have been passed\+: {\ttfamily -\/t 3 -\/n \char`\"{}\+My Name\char`\"{} -\/-\/enable} 
\begin{DoxyCode}
\{C++\}
// -t 3
parse<int>(<cmd>, <argument iterator position 0>, <cmdargs>)
// calls and returns the value of
convert<int>(<cmd>, <argument iterator position 1>, <cmdargs>)
// which is the integer 3.


// -n "My Name"
parse<std::string>(<cmd>, <argument iterator position 2>, <cmdargs>)
// calls and returns the value of
convert<std::string>(<cmd>, <argument iterator position 3>, <cmdargs>)
// which is the string "My Name".


// --enable
// flags are checked only for their existence inside <cmd>,
// they don't provide any conversion methods.
\end{DoxyCode}


\subsection*{Primitive numerical types}

Let\textquotesingle{}s look at the {\ttfamily int} type, for example\+: 
\begin{DoxyCode}
\{C++\}
template <>
struct argument<int>
\{
    static int parse(const types::CommandLineArgument\_t &cmdarg)
    \{
        return numerical\_argument::parse<int>(
            cmdarg,
            CPPARGPARSE\_NUMERICAL\_ARGUMENT\_CONVERTER\_OVERLOADS(std::stoi),
            "integer"
        );
    \}

    static int convert(const types::CommandLineArgument\_t &cmdarg)
    \{
        return numerical\_argument::convert<int>(
            cmdarg,
            CPPARGPARSE\_NUMERICAL\_ARGUMENT\_CONVERTER\_OVERLOADS(std::stoi),
            "integer"
        );
    \}
\};
\end{DoxyCode}


All it does is call {\ttfamily cppargparse\+::numerical\+\_\+argument\+::parse$<$int$>$()} (or {\ttfamily cppargparse\+::numerical\+\_\+argument\+::convert$<$int$>$()}) on a command line argument iterator position {\ttfamily cmdarg} and tell the former to use {\ttfamily std\+::stoi} as the conversion function.

{\ttfamily float}, {\ttfamily double}, and {\ttfamily long double} are implemented in the same way, using
\begin{DoxyItemize}
\item {\ttfamily std\+::stof},
\item {\ttfamily std\+::stod} and
\item {\ttfamily std\+::stold}
\end{DoxyItemize}

as conversion functions, respectively.

The {\ttfamily cppargparse\+::numerical\+\_\+argument} namespace provides the actual parser and converter functions\+: 
\begin{DoxyCode}
\{C++\}
template <typename T>
static T convert(
        const types::CommandLineArgument\_t &cmdarg,
        const std::function<T(const std::string &, size\_t *)> &numerical\_converter,
        const std::string &type\_string
    )
\{
    try
    \{
        return numerical\_converter(*cmdarg, 0);
    \}

    catch (std::invalid\_argument const &)
    \{
        throw errors::CommandLineOptionError(error\_message(cmdarg, type\_string));
    \}
    catch (std::out\_of\_range const &)
    \{
        throw errors::CommandLineOptionError(error\_message(cmdarg, type\_string));
    \}
\}

template <typename T>
static T parse(const types::CommandLineArgument\_t &cmdarg,
        const std::function<T(const std::string &, size\_t *)> &numerical\_converter,
        const std::string &type\_string)
\{
    return convert(std::next(cmdarg), numerical\_converter, type\_string);
\}
\end{DoxyCode}


{\ttfamily cppargparse\+::convert()} calls {\ttfamily numerical\+\_\+converter} with a string as the argument. {\ttfamily numerical\+\_\+converter} can (currently) be one of the following\+:
\begin{DoxyItemize}
\item {\ttfamily std\+::stoi}
\item {\ttfamily std\+::stof}
\item {\ttfamily std\+::stod}
\item {\ttfamily std\+::stold}
\end{DoxyItemize}

A conversion error is indicated by throwing {\ttfamily errors\+::\+Command\+Line\+Option\+Error} with a custom error message.

{\ttfamily cppargparse\+::parse()} calls {\ttfamily cppargparse\+::convert()} with the command line argument next in line.

\subsection*{String types}

Currently only {\ttfamily std\+::string} is implemented as a string type\+: 
\begin{DoxyCode}
\{C++\}
static const std::string convert(
        const types::CommandLine\_t &cmd,
        const types::CommandLineArgument\_t &cmdarg,
        const types::CommandLineArgumentsMap\_t &)
\{
    if (cmdarg == cmd.cend())
    \{
        throw errors::CommandLineOptionError(error\_message(cmdarg));
    \}

    return *cmdarg;
\}
\end{DoxyCode}


It just checks whether the end of the command line argumnents list has been reached. If so, thrown an error, if not, return its string value.

\subsection*{Vector types}

Currently only {\ttfamily std\+::vector$<$T$>$} is implemented as a vector type\+: 
\begin{DoxyCode}
\{C++\}
static const std::vector<T> parse(
        const types::CommandLine\_t &cmd,
        const types::CommandLineArgument\_t &cmdarg,
        const types::CommandLineArgumentsMap\_t &cmdargs)
\{
    auto options = get\_option\_values(cmd, cmdarg, cmdargs);
    std::vector<T> values;

    for (auto option : options)
    \{
        values.emplace\_back(argument<T>::convert(cmd, option, cmdargs));
    \}

    return values;
\}


static types::CommandLineArguments\_t get\_option\_values(
        const types::CommandLine\_t &cmd,
        const types::CommandLineArgument\_t &cmdarg,
        const types::CommandLineArgumentsMap\_t &cmdargs)
\{
    types::CommandLineArguments\_t values;

    for (auto cmdarg\_value = std::next(cmdarg); cmdarg\_value != cmd.end(); ++cmdarg\_value)
    \{
        if (cmdargs.find(*cmdarg\_value) != cmdargs.cend())
        \{
            break;
        \}

        values.emplace\_back(cmdarg\_value);
    \}

    return values;
\}
\end{DoxyCode}


It collects all values tied to the argument and converts each value string to {\ttfamily T}.

\subsection*{Custom types}

... todo ...

\section*{Build tests and install}

Build requirements\+:
\begin{DoxyItemize}
\item {\ttfamily gcc}, {\ttfamily clang}, or {\ttfamily msvc}
\item {\ttfamily cmake}
\end{DoxyItemize}

Then you should be able to build the tests like this\+: 
\begin{DoxyCode}
git clone https://github.com/backraw/cppargparse.git
cd cppargparse

mkdir build
cd build

cmake -DCMAKE\_BUILD\_TYPE=Profiling ..
make tests
\end{DoxyCode}


Of course, you can use every generator that C\+Make supports.

To install the library, run {\ttfamily sudo make install} inside the build directory. 